
\chapter*{Conclusion générale et perspectives}
\addcontentsline{toc}{chapter}{Conclusion générale}
\markboth{Conclusion}{Conclusion}
\label{sec:conclusion}
Notre projet intitulé « Conception et réalisation d’une pateforme d’E-learning » consiste à la conception et la réalisation d’une plateforme web destiné pour l’apprentissage en ligne.
\medskip
Contrairement à la majorité des travaux existants sur le marché qui offrent des fonctionnalités
limités et nécessitent un effort de configuration considérable, nous avons réalisé un système
qui permet à la fois de gérer des formations, simuler un salon de formation virtuel, et donner la possibilité aux formateurs de guider leurs apprenants .
\medskip
En ce qui concerne la démarche, nous avons en premier lieu 
commencé  par comprendre le contexte général de notre application et identifier les différentes exigences de notre futur système. Nous avons préparé par la suite notre planning de travail en respectant les priorités de nos besoins suite à une discussion entre l'équipe du développement et le directeur du produit.Tout au long de notre cycle de développement nous avons couplé la méthodologie Scrum par une autre méthodologie agile . En deuxième lieu nous avons spécifié notre application pour
discerner les fonctionnalités .En troisième lieu, nous avons procédé à sa conception ainsi
qu’aux choix technologiques pour sa réalisation. Enfin, nous l’avons mise en œuvre.
	\begin{figure}[ht]
	\centering
	\includegraphics[width=7cm,height=4cm]{Processusactueldedéveloppement.png}
	\caption{Processus actuel de développement.}
	\label{fig:Processus actuel de développement }
\end{figure}
\FloatBarrier
\medskip
En terme de perspectives, le projet dans sa version actuelle n’est pas encore achevé, on est actuellement en cours du personnalisation de l'extension backoffice pour facilité encore plus l'accès et la gestion des informations des produits. Les prochaines itération porteront toujours sur ce sujet en plus de l'évolution du processus d'éxport des produits pour qu'il soit beaucoup plus personnalisable par l'utilisateur.
\medskip
    




























  

%%% Local Variables: 
%%% mode: latex
%%% TeX-master: "isae-report-template"
%%% End: 

