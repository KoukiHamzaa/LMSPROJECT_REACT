\chapter*{Conclusion générale}
\addcontentsline{toc}{chapter}{Conclusion générale}
\markboth{Conclusion}{Conclusion}
\label{sec:conclusion}

    Mon projet de fin d’études au sein de SQLI consistait en la mise en œuvre d’une solution PIM en faveur d’une maison de produits de luxe à base de la plateforme e-commerce SAP Hybrise. La solution avait pour objectif de permettre au client de gérer les informations relatives a ses produits d’une façon centralisée et ce pour des raisons d'enrichissement et de diffusion vers les différents canaux comme les sites é-commerce, les points de vente, les applications mobiles, etc.
    
    \medskip
    
    La première itération du projet consistait en la définition du périmètre du projet. Les itérations qui suivaient ont été consacrées à la réalisation de plusieurs modules métiers de base comme le flux d'import et la gestion des profils. Lors de chaque itération, une étude fonctionnelle a été effectuée suivie d’une conception détaillée permettant par la suite d’enchaîner la phase de réalisation et de développement des modules en question.

    \medskip
    
  Le présent projet m’a apporté de la valeur ajouté sur plusieurs niveaux. D’une part, j’ai appris a mieux gérer mon temps, mieux communiquer et collaborer avec les autres membres de l’équipe de développement. D’autre part, il m’a permis d’acquérir des compétences extrêmement importantes telle que le maintien et l'optimisation de mon code, l'intégration des bonnes pratiques, l’autonomie ainsi qu’une grande capacité de modélisation.
  
      \medskip

    En terme de perspectives, le projet dans sa version actuelle n’est pas encore achevé, on est actuellement en cours du personnalisation de l'extension backoffice pour facilité encore plus l'accès et la gestion des informations des produits. Les prochaines itération porteront toujours sur ce sujet en plus de l'évolution du processus d'éxport des produits pour qu'il soit beaucoup plus personnalisable par l'utilisateur.
    
        \medskip
    
    % J’espère que le présent rapport a pu bien clarifier les éléments principaux de mon projet d’une manière simple et concise et qu’il a été un apport positif pour les gens l’ayant lus et jugés.

    

%%% Local Variables: 
%%% mode: latex
%%% TeX-master: "isae-report-template"
%%% End: 

