\chapter*{Introduction générale}
\addcontentsline{toc}{chapter}{Introduction}
\markboth{Introduction}{Introduction}
\label{chap:introduction}
%\minitoc



\medskip

L’enseignement est un mode d’éducation permettant de développer les connaissances d’un
élève par le biais de la communication verbale et écrite. Il est centré sur le cours magistral.
\medskip

Les systèmes traditionnels d’enseignement imposent à tous les apprenants une unité de lieu,
une unité de temps, une unité d’action, une unité de rythme ce qui implique une rigidité des
mécanismes et une difficulté d’adéquation avec la réalité quotidienne. La tendance à l’amélioration du système sur le plan pédagogique par le recours aux moyens audiovisuels classiques
(projections de diapositives, de transparents, séquences vidéo) n’a pas résolu le problème. Il
existe en effet une solution de rechange à l’enseignement traditionnel. Cette forme d’enseignement relativement jeune, c’est la formation à distance qui permet d’acquérir des connaissances
et de développer des habiletés sans avoir à fréquenter un établissement d’enseignement et sans
la présence physique d’une personne qui enseigne. Dans ce cadre s’intègre notre projet de fin
d’étude qui est effectué au sein de la société N3RD. Notre objectif est de concevoir et mettre
en place un système partagé et sécurisé dédié à l’enseignement à distance.

\medskip

Ce rapport s’articulera donc, autour de quatre chapitres comme suit :
\medskip

Le premier chapitre permet de placer notre projet dans son contexte générale. Il comportera
une description de l’organisme d’accueil, exposera l’étude de l’existant, mettra l’accent sur la
solution proposée et abordera la méthode adoptée. Il présentera également quelques notions
théoriques importantes pour la réalisation de notre projet.

\medskip

Dans le deuxième chapitre, nous dégagerons les besoins les besoins fonctionnels et non fonctionnels, comme nous spécifierons un diagramme de cas d’utilisation général du produit avec le
langage de modélisation unifié UML.

\medskip

Le troisième chapitre elaborera l’étude conceptuelle de notre projet dont lequel nous allons
presenter quelues diagrammes.

\medskip

Le quatrième chapitre « La phase de clôture » sert à présenter l’ensemble des différents outils
utilisés pour concevoir et développer notre application ainsi que le diagramme de déploiement
et nous terminons par des captures d’écrans des principaux interfaces de l’application pour
illustrer la version finale de notre produit.

\medskip

Finalement, nous clôturerons ce rapport par une conclusion générale dans laquelle nous évaluerons le travail réalisé au sein de la société et nous proposerons des perspectives dans le but
d’améliorer notre travail.
